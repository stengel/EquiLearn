% using Lemke
\documentclass[a4paper,12pt]{article}  %% important: a4paper first
%
\usepackage[notcite,notref]{showkeys}
\pdfoutput=1
\usepackage{natbib} 
\usepackage{amssymb}
\usepackage{amsthm}
\usepackage{newpxtext,newpxmath} 
\usepackage{microtype}
\linespread{1.10}        % Palatino needs more leading (space between lines)
\usepackage{xcolor}
\usepackage{pict2e} 
\usepackage{bimatrixgame}
\usepackage{tikz} 
\usetikzlibrary{shapes}
\usetikzlibrary{arrows.meta}
%\usepackage{smallsec}
\usepackage{graphicx}
%\usepackage[pdflatex]{hyperref}
\usepackage[hyphens]{url} 
\usepackage[colorlinks,linkcolor=blue,citecolor=blue]{hyperref}
%\usepackage{hyperref}
\urlstyle{sf}
\usepackage[format=hang,justification=justified,labelfont=bf,labelsep=quad]{caption} 
% \input macros-drawtree
\oddsidemargin=.46cm    % A4
\textwidth=15cm
\textheight=23.3cm
\topmargin=-1.3cm
\clubpenalty=10000
\widowpenalty=10000
\predisplaypenalty=1350
\sfcode`E=1000  % normal spacing if E followed by period, as in "EFCE."
\sfcode`P=1000  % normal spacing if P followed by period, as in "NP." 
\newdimen\einr
\einr1.7em
\newdimen\eeinr 
\eeinr 1.7\einr
\def\aabs#1{\par\hangafter=1\hangindent=\eeinr
    \noindent\hbox to\eeinr{\strut\hskip\einr#1\hfill}\ignorespaces}
\def\rmitem#1{\par\hangafter=1\hangindent=\einr
  \noindent\hbox to\einr{\ignorespaces#1\hfill}\ignorespaces} 
\newcommand\bullitem{\rmitem{\raise.17ex\hbox{\kern7pt\scriptsize$\bullet$}}} 
\def\subbull{\vskip-.8\parskip\aabs{\raise.2ex\hbox{\footnotesize$\circ$}}}
\let\sfield\mathcal
\newtheorem{theorem}{Theorem}
\newtheorem{corollary}[theorem]{Corollary}
\newtheorem{example}[theorem]{Example}
\newtheorem{lemma}[theorem]{Lemma}
\newtheorem{proposition}[theorem]{Proposition}
\theoremstyle{definition}
\newtheorem{remark}[theorem]{Remark}
\newtheorem{definition}[theorem]{Definition}
\def\reals{{\mathbb R}} 
\def\eps{\varepsilon}
\def\prob{\hbox{prob}}
\def\sign{\hbox{sign}}
\def\proof{\noindent{\em Proof.\enspace}}
\def\proofof#1{\noindent{\em Proof of #1.\enspace}}
\def\endproof{\hfill\strut\nobreak\hfill\tombstone\par\medbreak}
\def\tombstone{\hbox{\lower.4pt\vbox{\hrule\hbox{\vrule
  \kern7.6pt\vrule height7.6pt}\hrule}\kern.5pt}}
\def\eqalign#1{\,\vcenter{\openup.7ex\mathsurround=0pt
 \ialign{\strut\hfil$\displaystyle{##}$&$\displaystyle{{}##}$\hfil
 \crcr#1\crcr}}\,}
\def\zw#1\par{\vskip2ex{\textbf{#1}}\par\nobreak} 
\newdimen\pix  % bounding box height for eps files
\pix0.08ex
\newsavebox{\figA} 
\parindent0pt
\parskip1.3ex

\newcommand{\T}{^{\top}}
\newcommand{\0}{{\mathbf0}}
\newcommand{\1}{{\mathbf1}}


\title{%
Path-Following for Bimatrix Games Using Lemke's Algorithm
}
\author{Bernhard von Stengel}
\date{May 1, 2023
%\date{\today
\\[1ex]
}

\begin{document}
\maketitle

\begin{abstract}
We study three path-following methods that find at least one
equilibrium of a bimatrix game, and show how to implement
them as special cases of Lemke's algorithm.
These methods are the global Newton method, with the
Lemke-Howson algorithm as a special case, and the
van den Elzen-Talman linear tracing procedure.

% \noindent 
% \textbf{ACM classification:} 
% CCS
% $\to$ Theory of computation
% $\to$ Theory and algorithms for application domains
% $\to$ Algorithmic game theory and mechanism design
% $\to$ Solution concepts in game theory,
% exact and approximate computation of equilibria,
% representations of games and their complexity
% 
% 
% \strut
% 
% \noindent 
% \textbf{AMS subject classification:} 
% 91A18  (Games in extensive form)
% 
% \strut
% 
% \noindent 
% \textbf{JEL classification:} 
% C72 (Noncooperative Games)
% 
% \strut
% 
% \noindent 
% \textbf{Keywords:}
% extensive game,
% correlated equilibrium,
% polynomial-time algorithm,
% computational complexity.
% 
\end{abstract}

\section{Linear complementarity} 
\label{s-lcp}

All vectors are column vectors, and $\0$ and $\1$ are the
all-zero and all-one vector, respectively, their dimension
depending on the context.
Vectors and scalars are treated as matrices and positioned
such that matrix multiplication works (as in column vector
times a scalar, but scalar times a row vector).

The dimension of an LCP is denoted by $N$ rather than $n$
because we consider $m\times n$ bimatrix games.

The standard linear complementarity problem or LCP $(q,M)$
is specified by an $N$-vector $q$ and an $N\times N$ matrix
$M$ and states: find an $N$-vector $z$ such that
\begin{equation}
\label{lcp}
z\ge\0
\quad\bot\quad
q+Mz\ge\0
\end{equation}
where the inequalities are meant to hold and
the symbol $\bot$ denotes orthogonality and thus
complementarity of the respective slacks; that is, for
general $N$-vectors $a,b,c,d$:
\begin{equation}
\label{bot}
a\ge b
\quad\bot\quad 
c\ge d
\qquad:\Leftrightarrow\qquad
a\ge b,~~
c\ge d,~~
(a-b)\T (c-d)=0\,.
\end{equation}

In an $m\times n$ bimatrix game $(A,B)$, let 
$X$ and $Y$ be the mixed-strategy sets of the row and column
player, respectively,
\begin{equation}
\label{XY}
X=\{x\in\reals^m\mid x\ge\0,~\1\T x=1\},
\qquad
Y=\{y\in\reals^n\mid y\ge\0,~\1\T y=1\}. 
\end{equation}
A mixed Nash equilibrium or just \textit{equilibrium} of
$(A,B)$ is a pair $(x,y)\in X\times Y$ such that there are
reals $u$ and $v$ with
\begin{equation}
\label{NE}
x\ge\0
\quad\bot\quad 
Ay\le\1 u\,,\qquad
y\ge\0
\quad\bot\quad 
B\T x\le\1 v\,,
\end{equation}
where $u$ and $v$ are the equilibrium payoffs to the row and
column player, respectively.
Condition (\ref{NE}) is the best-response condition that
every pure strategy that has positive probability must be a
pure best response against the other player's mixed
strategy \citep[proposition 6.1]{vS22}.

This is a special case of an LCP, and the algorithm 
by \citet{Lemke1965} for solving an LCP can be used to
represent several path-following methods for finding an
equilibrium of a bimatrix game.
In order to do so, we will employ several variable
transformations.

First, we represent the unrestricted
variables $u$ and $v$ in (\ref{NE}) as differences of
nonnegative variables in the form
\begin{equation}
\label{uv+-}
u=u^+-u^-,\quad u^+\ge0,~ u^-\ge0,~
\qquad
v=v^+-v^-,\quad v^+\ge0,~ v^-\ge0,~ 
\end{equation}
and the equations $\1\T x=1$ and $\1\T y=1$ as two
inequalities, as follows:
\begin{equation}
\label{lcpAB}
\arraycolsep.2em
\begin{array}{rcrrllcrr}
% \begin{array}{rcllrrcrr} 
x\ge\0&\quad\bot\quad&
&-Ay
&
{}+\1u^+-\1u^-
&&\ge&\0
\\
y\ge\0&\quad\bot\quad&
-B\T x&&
&{}+\1v^+-\1v^-
&\ge&\0&
\\
u^+\ge0&\quad\bot\quad&
-\1\T x&&&&\ge&-1
\\
u^-\ge0&\quad\bot\quad&
\1\T x&&&&\ge&1
\\
v^+\ge0&\quad\bot\quad&
&-\1\T y&&&\ge&-1
\\
v^-\ge0&\quad\bot\quad&
&\1\T y&&&\ge&1
\end{array}
\end{equation}
Suppose that neither equilibrium payoff $u$ or $v$ is zero.
For example, if $v<0$ then $v^->0$ and the last inequality
in (\ref{lcpAB}) is tight, that is, $\1\T y=1$ and thus
$y\in Y$.
We have chosen the complementarity conditions in
(\ref{lcpAB}) so that the constraint matrix $M$ is
skew-symmetric ($M=-M\T$) if the game is zero-sum
($B=-A$). 
It will be convenient to have an LCP matrix $M$ where the
sub-matrices $-A$ and $-B\T$ are positive, that is,
\begin{equation}
\label{ABneg}
A<0,\qquad B<0,
\end{equation}
which can always be achieved by subtracting a sufficiently
large constant from the entries of $A$ and $B$.
Then any equilibrium payoffs $u$ and $v$ are negative, so
that we can choose
\begin{equation}
\label{uv-}
u^+=v^+=0,\qquad
u=-u^-,\qquad
v=-v^-,
\end{equation}
and (\ref{lcpAB}) simplifies to
\begin{equation}
\label{lcpAB-}
\arraycolsep.2em
\begin{array}{rcrrllcrr}
% \begin{array}{rcllrrcrr} 
x\ge\0&\quad\bot\quad&
&-Ay
&
{}-\1u^-
&&\ge&\0
\\
y\ge\0&\quad\bot\quad&
-B\T x&&
&{}-\1v^-
&\ge&\0&
\\
u^-\ge0&\quad\bot\quad&
\1\T x&&&&\ge&1
\\
v^-\ge0&\quad\bot\quad&
&\1\T y&&&\ge&1&.
\end{array}
\end{equation}
% \begin{equation}
% \label{lcpAB-}
% \arraycolsep.2em
% \begin{array}{rcllrrcrr}
% u^-\ge0&\quad\bot\quad&
% &&\1\T x&&\ge&1
% \\
% v^-\ge0&\quad\bot\quad&
% &&&\1\T y&\ge&1
% \\
% x\ge\0&\quad\bot\quad&
% -\1u^-
% &&&-Ay
% &\ge&\0
% \\
% y\ge\0&\quad\bot\quad&
% &-\1v^-
% &-B\T x&
% &\ge&\0&.
% \end{array}
% \end{equation}
By (\ref{ABneg}) and complementarity, $u^-=-u>0$ and
%$v^-=-v>0$ in any solution $(u^-,v^-,x,y)$ to 
$v^-=-v>0$ in any solution $(x,y,u^-,v^-)$ to
% (\ref{lcpAB-}), and the first two inequalities are therefore
(\ref{lcpAB-}), and the last two inequalities are therefore
tight, that is, $x\in X$ and $y\in Y$. 

A second variable transformation changes the ``homotopy
parameter'' $\lambda$ in the ``global Newton method'' (see
Section~\ref{s-path}). 
We first summarize Lemke's algorithm.

\section{Lemke's algorithm}
\label{s-lemke}

\citet{Lemke1965} described an algorithm for solving the LCP
(\ref{lcp}).
It uses an additional $N$-vector $d$, called
\textit{covering vector}, with a corresponding scalar variable~$z_0$,
and computes with \textit{basic solutions} to the augmented system
\begin{equation}
\label{lcp0}
z\ge\0,\quad z_0\ge0,\qquad
w=q+Mz+dz_0\ge\0,\qquad
z\T w=0\,.
\end{equation}
Any solution $z,z_0$ to (\ref{lcp0}) is called
\textit{almost complementary}, and
\textit{complementary} if $z_0=0$, which implies (\ref{lcp}).
The vector $d$ is assumed to fulfill
\begin{equation}
\label{d}
d\ge\0,
\qquad
q_i<0\quad\Rightarrow\quad d_i>0
\quad
(1\le i\le N). 
\end{equation}
This implies that for $z=\0$ and all sufficiently large
$z_0$ we have $w = q + dz_0\ge\0$ and $z\T w = 0$, so that
(\ref{lcp0}) holds.
The set of these almost complementary solutions is called
the \textit{primary ray}.

For initialization, let $z_0$ be minimal such that $w=q+dz_0\ge\0$.
Unless $q\ge\0$ (in which case the LCP is solved
immediately), $z_0$ is positive and some component $w_i$ of
$w$ is zero.
A first pivoting step lets $z_0$ enter and $w_i$ leave the
basis.
The algorithm then performs a sequence of
\textit{complementary pivoting} steps.
At each step, one variable of a complementary pair $(z_i,w_i)$
leaves and then its \textit{complement} (the other variable)
enters the basis, which maintains the condition $z\T w=0$.
With a suitable lexicographic symbolic perturbation, the
leaving variable (including the first leaving variable $w_i$
when $z_0$ enters) is always unique, so that the algorithm
follows a unique path.
% In a mixed LCP, a variable $z_i$ without sign restrictions
% never leaves the basis.
The goal is that eventually $z_0$ leaves the basis and then
has value zero, so that the LCP (\ref{lcp}) is solved.

Lemke's algorithm could also terminate with a
\textit{secondary ray}, which happens when the entering
variable can assume arbitrarily positive values (as in the
simplex algorithm for linear programming for an unbounded
objective function).
Certain conditions on the LCP data $(q,M)$ exclude this ray
termination.
These hold if $A$ and $B$ are negative as in (\ref{ABneg}),
as shown in \citet[theorem 4.1]{KMvS96} if the LCP is
presented in its original form (\ref{lcpAB}), and also in
its shortened form (\ref{lcpAB-}) using additional
considerations.
We will be able to exclude ray termination with a more
direct argument.

% \begin{theorem}
% \label{t-noray}
% If $z\T Mz\ge0$ for all $z\ge\0$, and $z\ge\0$, $Mz\ge\0$
% and $z\T Mz=0$ imply $z\T q\ge0$, then Lemke's algorithm
% computes a solution of the LCP $(\ref{lcp})$ and does not
% terminate with a secondary ray.
% \end{theorem}
% 
% \proof
% See \citet[theorem 4.4.13]{CPS} or
% \citet[theorem 4.1]{KMvS96}, which has detailed comments on related
% observations, including in \cite{Lemke1965}.
% \endproof
% 
% \begin{theorem}
% With negative payoff matrices $A$ and $B$ as in
% $(\ref{ABneg})$ and, as in $(\ref{lcpAB-})$,
% \begin{equation}
% \label{tlcp}
% z=\left[\begin{matrix}x\\y\\ u^-\\v^-
% \end{matrix}\right],\quad
% q=\left[\begin{matrix}\0\\ \0\\  -1\\-1
% \end{matrix}\right],\quad 
% M=\left[\begin{matrix}
% &-A&\1\\
% -B\T&&&-\1\\
% \1\T &\\
% &\1\T\\
% \end{matrix}\right],\quad 
% \end{equation}

% the LCP $(q,M)$ fulfills the assumptions in
% Theorem~\ref{t-noray}.
% \end{theorem}
% 
% \proof
% We have $z\T Mz=-x\T Ay -y\T B\T x$, which by (\ref{ABneg})
% is nonnegative if $z\ge\0$.
% Suppose $z\ge\0~\bot~Mz\ge\0$, where we want to show that
% $z\T q=-u^--v^-\ge0$.
% We have $x\ge\0$,
% $y\ge\0$, $u^-\ge0$, $v^-\ge0$, $-Ay-\1u^-\ge\0$,
% $-B\T x-\1v^-\ge\0$, and
% $u^-\ge0~\bot~\1\T x\ge\0$ and
% $v^-\ge0~\bot~\1\T y\ge\0$.
% The latter condition implies $u^-=0$ if $x\ne\0$ and 
% \endproof 
% 
\section{Path-following of equilibria}
\label{s-path}

The general ``homotopy'' approach to finding an equilibrium
of a game in strategic form is to modify the game to a
one-parameter set of games with a real parameter~$\lambda$
such that $\lambda=0$ corresponds to the given game, and a
sufficiently large value of~$\lambda$ has an equilibrium
that is easy to find, which serves as a \textit{starting
point} of the algorithm.
Subsequently, $\lambda$ is lowered (with possible
intermittent increases) while ``tracing'' the equilibrium of
the parameterized game, until $\lambda=0$ and an equilibrium
of the given game is found.
The result is a path of parameterized games and a 
corresponding equilibrium for each game.
For a bimatrix game $(A,B)$ and the methods considered here,
the path is piecewise linear.
Up to scaling, it consists of line segments in the set
$X\times Y$ of mixed-strategy pairs, represented by the
pivoting steps of Lemke's algorithm.

For brevity, we first summarize the different methods and how to
represent them with Lemke's algorithm, with details in the
following subsections.
We always assume negative payoff matrices $A$ and $B$ as in
$(\ref{ABneg})$ and consider the LCP (\ref{lcp}) as
in~$(\ref{lcpAB-})$ with $N=m+n+2$ and
\begin{equation}
\label{tlcp}
z=\left[\begin{matrix}x\\y\\ u^-\\v^-
\end{matrix}\right],\qquad
q=\left[\begin{matrix}\0\\ \0\\  -1\\-1
\end{matrix}\right],\qquad 
M=\left[\begin{matrix}
&-A&-\1\\
-B\T&&&-\1\\
\1\T &\\
&\1\T\\
\end{matrix}\right].
\end{equation}
In the augmented LCP (\ref{lcp0}), the covering vector $d$
is chosen according to the different methods.
\bullitem
The \textit{global Newton method}, with subsidy vectors
$a\in\reals^m$ and $b\in\reals^n$ for the rows and columns,
traces equilibria of 
\begin{equation}
\label{ABl}
\Gamma_\lambda=(A + a\lambda\1\T, ~B+\1\lambda b\T)\,.
\end{equation}
This is represented in Lemke's algorithm with
\begin{equation}
\label{gnm}
d=\left[\begin{matrix} \1\alpha-a\\ \1\beta-b\\1\\1\\
\end{matrix}\right]
,\qquad 
z_0=\frac{\lambda}{1+\lambda}=\frac{1}{1/\lambda+1}
,\qquad 
\lambda = \frac{z_0}{1-z_0}
\end{equation}
with sufficiently large constants $\alpha$ and $\beta$ such
that $\1\alpha-a$ and $\1\beta-b$ are positive.
The parameters $\lambda\in[0,\infty]$ and $z_0\in[0,1]$ are
in a monotonic bijection via~(\ref{gnm}).

\bullitem
The \textit{Lemke-Howson (LH)} method uses a \textit{missing
label} $k$ in $\{1,\ldots,m+n\}$ for a row or column
strategy that is is subsidized. 
This is represented with $d\in\reals^N$ defined by
\begin{equation}
\label{LHd}
d_i=1\quad
(1\le i\le N,~i\ne k),\qquad
d_k=0\,.
\end{equation}
This has some disadvantages compared to the standard
implementation of the LH method, as will be discussed, but
maps nicely to Lemke's algorithm.

\bullitem
The \textit{tracing procedure} by van den Elzen and Talman
uses a \textit{prior} or starting point $(\bar x,\bar y)\in
X\times Y$.
It traces equilibria $(\hat x,\hat y)$ of a game where the
players play with weight (or probability) $\tau\in[0,1]$
against the prior and with weight $1-\tau$ against their
actual strategies $(\hat x,\hat y)$, starting with $\tau=1$
and ending with $\tau=0$.
This is represented with 
\begin{equation}
\label{ET}
d=\left[\begin{matrix}-A\bar y\\ -B\T\bar x \\ \1\\ \1
\end{matrix}\right],\qquad 
z_0=\tau,
\qquad
x=\hat x(1-z_0),
\quad
y=\hat y(1-z_0)
\end{equation}
with $z=(x,y,u^-,v^-)\T$ as in (\ref{tlcp}).

We now consider these methods in detail.

\subsection{The global Newton method}
\label{s-gnm}

The global Newton method \citep{GW03} works for any finite
game in strategic form, but we consider it here for two
players for an $m\times n$ bimatrix game $(A,B)$.
Let $a\in\reals^m$ and $b\in\reals^n$ where $a_i$ and $b_j$
are considered a ``subsidy'' (or penalty if negative) for
row $i$ and column $j$, respectively.
These subsidies are scaled simultaneously with $\lambda$ and
added to each respective row and column, which defines the
parameterized game (\ref{ABl}).

If $a$ and $b$ have a unique maximum $a_i$ and $b_j$, say,
then for sufficiently large $\lambda$ the game
$\Gamma_\lambda$ will have the pure-strategy pair $(i,j)$ as
its unique equilibrium as the starting point.
Lowering $\lambda$ will then either lead to $\lambda=0$ and
having that same equilibrium as an equilibrium of $(A,B)$,
or for some $\bar\lambda>0$ give a new pure strategy $k$ of
either player as a new best response.
At that point, the game $\Gamma_{\bar\lambda}$ is
necessarily degenerate, but has generically only a line segment,
at this initial step some convex combinations of $i$ and $k$
(if $k$ belongs to the row player), as an equilibrium
component.
Traversing that equilibrium component in
$\Gamma_{\bar\lambda}$ then may either lead to a new best
response, or dropping a previous best response (like~$i$),
which allows to change $\lambda$ again (normally a decrease,
but possibly also an increase of $\lambda$).

Generically, the encountered equilibrium components are
one-dimensional, and no bifurcation occurs;
when implemented with Lemke's algorithm, lexicographic
degeneracy resolution creates a unique path, even if the
maximum entries of $a$ and~$b$ are not unique.
The computed path consists of games~$\Gamma_\lambda$, each
with an associated unique equilibrium, which in case of
intermittent increases of~$\lambda$ depends on the path
history.
The path terminates with $\lambda=0$ because for
large~$\lambda$ the game $\Gamma_\lambda$ has a unique
equilibrium, so the path cannot continue with $\lambda$
becoming infinite (which would be a secondary ray in Lemke's
algorithm).

We now explain (\ref{gnm}).
The constants $\alpha$ and $\beta$ are just added to all
entries of $A$ and~$B$, respectively, and do not change best
responses.
For the game $\Gamma_\lambda$ in (\ref{ABl}), the
conditions for an equilibrium $(x,y)$ state
\begin{equation}
\label{Geq}
\arraycolsep.2em
\begin{array}{rcrrllrrr}
x\ge\0&\quad\bot\quad&
&-Ay
&
{}-\1u^-
&&{}-a\lambda\ge&\0
\\
y\ge\0&\quad\bot\quad&
-B\T x&&
&{}-\1v^-
&{}-b\lambda\ge&\0&
\\
u^-\ge0&\quad\bot\quad&
\1\T x&&&&\ge&1
\\
v^-\ge0&\quad\bot\quad&
&\1\T y&&&\ge&1&.
\end{array}
\end{equation}
Assuming that sufficiently large constants $\1\alpha$ and
$\1\beta$ are added to $-a$ and $-b$, this would work with
Lemke's algorithm with $z_0=\lambda$ and
$d=(\1\alpha-a,\1\beta-b,0,0)\T$ if $u^-$ and $v^-$
could be pivoted into the basis before $z_0$.
However, this is not a standard initialization where $z_0$
enters first, which requires condition (\ref{d}) to hold,
which fails here because $q=(\0,\0,-1,-1)\T$.
Instead, we divide all $m+n$ inequalities in
(\ref{Geq}) by $1+\lambda$, which is always positive,
and let 
\begin{equation}
\label{hatxy}
\textstyle
\hat x= x\frac1{1+\lambda},\quad 
\hat y= y\frac1{1+\lambda},\quad 
\hat u= {u^-}\frac1{1+\lambda},\quad 
\hat v= {v^-}\frac1{1+\lambda},\quad 
\tau= \frac\lambda{1+\lambda},\quad
1-\tau= \frac1{1+\lambda},
\end{equation}
so that $\1x\ge1$ becomes $\1\hat x\ge1-\tau$ and 
$\1y\ge1$ becomes $\1\hat y\ge1-\tau$, and (\ref{Geq}) 
becomes
\begin{equation}
\label{Geqnew}
\arraycolsep.2em
\begin{array}{rcrrllrrrr}
\hat x\ge\0&\quad\bot\quad&
&-A\hat y
&
{}-\1\hat u
&&{}-&a\tau\ge&\0
\\
\hat y\ge\0&\quad\bot\quad&
-B\T\hat x&&
&{}-\1\hat v
&{}-&b\tau\ge&\0&
\\
\hat u\ge0&\quad\bot\quad&
\1\T\hat  x&&&&+&\tau\ge&1
\\
\hat v\ge0&\quad\bot\quad&
&\1\T \hat y&&&+&\tau\ge&1&.
\end{array}
\end{equation}
With $-a$ replaced by $\1\alpha-a$,
and $-b$ replaced by $\1\beta-b$,
and $z_0=\tau$, this is the system (\ref{gnm}) apart
from the naming of the variables $\hat x,\hat y,\hat u,\hat v$,
with the standard initialization of Lemke's algorithm.
The variable $z_0$ enters the basis with value $z_0=1$, which
corresponds to $\lambda=\infty$.
The resulting basis is degenerate, and lexicographic
perturbation implies that $\hat v$ enters the basis next,
and $\hat u$ also has to enter before $z_0$ can be reduced.
However, the resulting path is unique and cannot terminate
with $z_0=1$ (or $\lambda=\infty$) because this would mean
an alternative way to start (a secondary ray is different
from the primary ray). 

During the initialization phase when $z_0=1$, we have
$\hat x=\0$ and $\hat y=\0$, but during the main computation
when $z_0<1$, the pair $(\hat x,\hat y)$ multiplied by
$\frac1{1-z_0}$ (that is, by $1+\lambda$) is the equilibrium
of the parameterized game $\Gamma_\lambda$.
When $z_0=0$ and the algorithm terminates, then $(\hat
x,\hat y)$ is an (unscaled) mixed equilibrium of the
original game $\Gamma_0$.

\subsection{The Lemke-Howson algorithm}
\label{s-lh}

% The Lemke-Howson algorithm \citep{LH}
The algorithm by \citet{LH}
has a nice visualization due to \cite{Shapley1974}
(see also \citealp{vS22}, chapter~9).
We number the $m+n$ pairs of inequalities in (\ref{NE}) with
$1,\ldots,m$ for the $m$ pure strategies of the row player,
and with $m+1,\ldots,m+n$ for the $n$ pure strategies of the
column player. 
Each number in $\{1,\ldots,m+n\}$ is called a
\textit{label}, and any $(x,y,u,v)$ in
$X\times Y\times\reals\times\reals$ is said
to have label $k$ if one of the corresponding inequalities
is tight; we also say $(x,y)$ has such a label if that is
the case for suitable $u$ and $v$. 
That is, $(x,y)$ has label $i$ in $\{1,\ldots,m\}$ if
$x_i=0$ or $(Ay)_i=u$, assuming $Ay\le\1u$, and has
label~$j$ in $\{m+1,\ldots,m+n\}$ if $y_j=0$ or
$(B\T x)_j=v$, assuming $B\T x\le\1v$.
In other words, $(x,y)$ has a pure strategy of either player
as a label if that pure strategy is played with probability
zero or is a best response.
An equilibrium $(x,y)$ is characterized by having all labels
$1,\ldots,m+n$.

The Lemke-Howson algorithm generalizes the equilibrium
condition by allowing one label, say~$k$, to be
\textit{missing}, and considers all $(x,y,u,v)$
in $X\times Y\times\reals\times\reals$
such that $Ay\le\1u$ and $B\T x\le\1v$
that have at least the labels in $\{1,\ldots,m+n\}-\{k\}$.
These tuples are called \textit{$k$-almost complementary},
and for a nondegenerate game define a collection of line
segments and one ray that form a set of paths and cycles.
The endpoints of the paths are the equilibria of $(A,B)$.
The ray is given by the pure strategy $k$ and its (pure)
best response, where in the system (\ref{NE}) the
constraints $Ay\le\1u$ for the payoff
variable $u$ if $k\in\{1,\ldots,m\}$, or the constraints
$B\T x\le\1v$ for the payoff variable~$v$ if
$k\in\{m+1,\ldots,m+n\}$, need not be tight
% in the system $Ay\le\1u$ (respectively, $B\T x\le\1v$)
because $k$ is not required to be a best response.
This works for the initial solutions given by the ray
because $k$ is the only pure strategy of the respective
player that has positive probability, and the other player
uses the (generically unique) pure best response to~$k$;
hence, all labels except $k$ are present.

The Lemke-Howson algorithm is a special case of the global
Newton method by setting the pair $(a,b)$ of ``subsidies''
in (\ref{ABl}) to be equal to the $k$th unit vector $e_k$ in
$\reals^{m+n}$ (that is, the $k$th component of $e_k$ is~1
and all other components are~0).
That way, the pure strategy $k$ is ``subsidized'' via
$\lambda$ so that it is a best response in $\Gamma_\lambda$
(and all other pure strategies are in equilibrium) but not
in the original game $(A,B)$. 
The algorithm ends with $\lambda=0$ because either $k$ has
probability zero or becomes a best response in $\Gamma_0$,
that is, in $(A,B)$.

With $(a,b)$ as the unit vector $e_k$ for the missing
label~$k$, we can choose $\alpha=\beta=1$ in (\ref{gnm}),
which gives (\ref{LHd}).
When running Lemke's algorithm this way, it indeed emulates
the LH method, where during the LCP computation the LCP
variables $x$ and $y$ in (\ref{tlcp}) have their components
sum to $1-z_0$ and need to be re-scaled to represent the
steps of the LH algorithm.
This is not possible in an initial phase where $z_0=1$ (and
thus $x=\0$ and $y=\0$, even though some components of $x$
and $y$ may already be basic.
This initial phase requires a lot (typically $m+n$)
degenerate pivots because the vector $d$ together with the
column entries for $u^-$ and~$v^-$ generates a large number
of basic variables with value zero.
As long as these are potential leaving variables, these need
to be pivoted out before $z_0$ can shrink in value.

A standard implementation of the LH algorithm uses
two best-response polytopes $P=\{x\in\reals^m\mid
x\ge\0,~B\T x\le\1\}$ and $Q=\{y\in\reals^n\mid Ay\le
\1,~y\ge\0\}$ with positive payoff matrices $A$ and $B$ (see
\citealp{vS22}, section 9.6f), with $(\0,\0)$ as an initial
``artificial'' equilibrium that is completely labeled.
Then pivoting alternates between two disjoint systems, one
for $x$ and one for $y$, which means that the LCP tableau is
always only half full.
In the implementation described here, the tableau is during
the intermittent computations three-quarters full.
It has, however, the advantage that only the LCP parameters
for the standard algorithm by Lemke have to be adapted, not
the algorithm itself. 

\subsection{The van den Elzen-Talman linear tracing procedure}
\label{s-et}

The algorithm by \citet{vdET99}
is another path-following method of ``tracing'' equilibria
in a certain restricted game, which is defined via a
``prior'' or starting point $(\bar x,\bar y)$ in the
mixed-strategy space $X\times Y$.
For $\tau\in[0,1]$, consider the restricted set of strategy
profiles
\begin{equation}
\label{Stau}
\arraycolsep.2em
\begin{array}{rcl}
S(\bar x,\bar y,\tau)
&=&\{\, (x,y)\in X\times Y\mid
x\ge\bar x\tau,~y\ge\bar y\tau\} 
\\
&=&\{\,(\hat x+\bar x\tau,\hat y+\bar y\tau)\mid
\1\T \hat x=1-\tau,~ \hat x\ge\0,~\1\T \hat y=1-\tau,~ \hat y\ge\0 \}
\\
&=&\{\, (\tilde x(1-\tau)+\bar x\tau,\tilde y(1-\tau)+\bar y\tau)\mid
(\tilde x,\tilde y)\in X\times Y\}
\,.
\end{array}
\end{equation}
For $(x,y)$ in the set $S(\bar x,\bar y,\tau)$,
every mixed-strategy probability $x_i$ or
$y_j$ has to be at least $\bar x_i$ or $\bar y_j$,
respectively. 
For $\tau=1$, clearly $S(\bar x,\bar y,1)=\{(\bar x,\bar y)\}$.
In the second equation in (\ref{Stau}),
$\hat x=x-\bar x\tau$ and $\hat y=y-\bar y\tau$
(we will use $\hat x$ and $\hat y$ as LCP variables), and in
the last equation of (\ref{Stau}) the mixed strategies
$\tilde x$ and $\tilde y$ are arbitrary for $\tau=1$ and
uniquely defined by $\tilde x=x\frac1{1-\tau}$
and $\tilde y=y\frac1{1-\tau}$ if $0\le \tau<1$.

The ``linear tracing procedure'' of \citet{vdET99}
follows a path of equilibria $(x, y)$
restricted to the strategy set $S(\bar x, \bar y, \tau)$,
parameterized by~$\tau$.
It starts with $\tau=1$ at the prior $(\bar x,\bar y)$,
which generically has a pair $(\tilde x,\tilde y)$ of pure best
responses.
The corresponding convex combination 
$(\tilde x(1-\tau)+\bar x\tau,\tilde y(1-\tau)+\bar y\tau)$
stays a best response against itself (i.e., is an
equilibrium) for $\tau$ slightly less than~1.
Subsequent lowering of $\tau$ either stops at $\tau=0$ or
for some $\bar\tau$ introduces a new best response
of some player which is then increased, with $\tau$ staying
fixed, until $\tau$ can be changed again.
Throughout, the path as represented in (\ref{Stau}) stays
inside the strategy space $(X,Y)$, and (in the generic case)
will not bifurcate or return to $(\bar x,\bar y)$ but will
eventually terminate at $\tau=0$ with an equilibrium of the
original game $(A,B)$. 

The interpretation of the linear tracing procedure is that
the prior represents a preconception of the players of what
their opponent will do.
The players reply first (for $\tau=1$) only to that
preconception, and then gradually, with weight $1-\tau$,
take their actual play $(\tilde x,\tilde y)$ into account, by
reacting to the respective convex combination with the
prior which has weight~$\tau$, adjusting their actions until
$\tau=0$.
Possible intermittent increases of $\tau$ in this adjustment
process are possible.

\citet{balthasar2010} has shown, and it is easy to see, that
the path of the linear tracing procedure for $0\le\tau<1$ is
in one-to-one correspondence with the path for the global
Newton method via
\begin{equation}
\label{Ntra}
\lambda=\frac{\tau}{1-\tau}=\frac1{1/\tau-1},\qquad
(a,b)=(A\bar y, B\T \bar x)
\end{equation}
in (\ref{ABl}), where $\lambda\in[0,\infty)$ and
$\tau\in[0,1)$ are strictly monotonic functions of each
other.
The starting point $(\bar x,\bar y)$ of the linear tracing
procedure corresponds to the ray of equilibria of
$\Gamma_\lambda$ when $\lambda\to\infty$ as $\tau\to1$.
We have used exactly this transformation in (\ref{hatxy}) to
use the parameter $\tau$ instead of $\lambda$ for the global
Newton method.

\citet{balthasar2010} has also shown that there are open
sets of games where an equilibrium of positive index is
\textit{not} found via the linear tracing procedure.
For example, the symmetric $3\times 3$ game $(A,B)$
\begin{equation}
\label{33}
A=B\T=\left[\,\begin{matrix}
~~0 & -1 & -c\\
-c & ~~0 & -1\\
-1 & -c & ~~0\\
\end{matrix}\,\right] 
\end{equation}
has a completely mixed equilibrium, which has positive
index, which for $1<c<2$ is not found for any generic prior
$(\bar x,\bar y)$.
Condition (\ref{Ntra}) shows that the global Newton method
has more general parameters $(a,b)$ than the linear tracing
procedure.
% even if both $A$ and $B\T$ have full rank (which
% requires $m=n$), as in (\ref{33}), for any~$a$ there is a
% solution $\bar y\in Y$ so that $a=A\bar y$ up to a scaling
% factor, but for an arbitrary $b$ the equation $b=B\T\bar x$
% may require a different scalar factor so that $\bar x\in X$.
For example, via $\lambda$ in (\ref{ABl}) the strategies of
the two players may vary by different ``speeds'' of change
for a given direction $(a,b)$, which cannot be emulated by a
suitable prior $(\bar x,\bar y)$.
It is open whether the completely mixed equilibrium in
(\ref{33}) (or in general any equilibrium of positive index)
can be found with the global Newton method.

Clearly, the linear tracing procedure is implemented via 
Lemke's algorithm by (\ref{ET}), using $a=A\bar y$ and
$b=B\T\bar x$ in (\ref{Geqnew}).
It would be of interest if it can serve as a good substitute
for an evolutionary dynamics (like the replicator dynamics).
One first question in this direction is whether the
completely mixed equilibrium of the game in (\ref{33}) is
dynamically stable in an evolutionary setting.

\section{Random starting points}

One way of using the linear tracing procedure is for
equilibrium selection.
We propose to choose starting vectors $(\bar x,\bar y)$
where $\bar x$ and $\bar y$ are each chosen from a uniform
distribution on their respective mixed-strategy simplices.
The corresponding percentages of which equilibrium is found
is then an indication of the ``prevalence'' of this
equilibrium, in particular a ``practical'' uniqueness if
the found equilibrium is always the same.

[Details on implementation.]

% \cite{vS2010}

\small
\bibliographystyle{book}
\bibliography{bib-evol} 

\end{document}

